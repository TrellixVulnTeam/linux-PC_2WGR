% !TEX TS-program = xelatex
% !TEX encoding = UTF-8 Unicode
% !Mode:: "TeX:UTF-8"

\documentclass{resume}
\usepackage{zh_CN-Adobefonts_external} % Simplified Chinese Support using external fonts (./fonts/zh_CN-Adobe/)
% \usepackage{NotoSansSC_external}
% \usepackage{NotoSerifCJKsc_external}
% \usepackage{zh_CN-Adobefonts_internal} % Simplified Chinese Support using system fonts
\usepackage{linespacing_fix} % disable extra space before next section
\usepackage{cite}

\begin{document}
\pagenumbering{gobble} % suppress displaying page number

\name{毕成}

\basicInfo{
	\email{bc970321@163.com} \textperiodcentered\ 
	\phone{(+86) 15765304024} \textperiodcentered\ 
	\github[bc970321]{https://github.com/bc970321}
}

\section{\faGraduationCap\ 教育背景}
\datedsubsection{\textbf{佳木斯大学}, 佳木斯}{2016 -- 至今}
\textit{在读本科生}\ 计算机科学与技术, 预计 2020 年 7 月毕业。

\section{\faUsers\ 实习经历}


\datedsubsection{\textbf{北京数美时代科技有限公司}}{2019年11月 -- 2020年2月}
\role{Golang研发实习生}{平台架构部}
\begin{onehalfspacing}
	\begin{itemize}
		\item 负责对天网RE层的功能研发
		\\~ 接入统计服务模块、新增手机号Md5转换、调整deviceId校验方式。
		\item 负责新增天网事件的上线以及产品API文档的更新维护
		\\~ 天网加事件以及相应事件的API接口文档的更新。
		\item 参与整体框架的重构
		\\~ 设计天网V4版本与接入历史记录的技术方案。
		\item 实习收获
		\\~ 了解并学习了高并发、快速响应的工程项目框架的设计,熟悉了后端开发相关的任务研发流程。

		

	\end{itemize}
\end{onehalfspacing}

\section{\faUsers\ 项目经历}


\datedsubsection{\textbf{分布式服务器健康监控系统}}{2019年3月 -- 2019年5月}
\role{C, Linux}{个人项目}
\begin{onehalfspacing}
	\begin{itemize}
		\item 项目简介:
		\\~ 基于socket和tcp协议的分布式服务器健康监控系统,采用CS结构实现由服务器对大量客户端进行健康监控的功能;
		\item 项目目标:
		\\~ Server+Client的结构,支持上万Client并发,Client端监控集群硬件信息,CPU,内存,磁盘等资源消耗,用户及进程等系统信息,并按照各个指标进行健康状态分级,将故障和可能的风险对Server报警,并将健康信息发送到Server。
		\item Client端功能介绍: 
		\\~ Client端使用多进程提高并发度,通过TCP协议进行安全的数据传输,发送心跳信息和等待Server端的链接请求。考虑服务器故障因素,将进行自我检测服务器心跳。考虑磁盘占用量,将大文件进行压缩,使用UDP与服务器进行警告信息通讯。
		\item Server端功能介绍
		\\~ Server端使用多线程处理各节点数据,监测在线用户、使用epoll主动心跳,维护链表信息,对掉线的Client端进行删除重连、发送数据请求信息并接收数据。使用IO多路复用,提升系统效率。设置警报端口,及时接受警报信息。
		\item 健康信息的获取:
		\\~ 使用系统调用的方法获取各硬件信息,CPU,内存,磁盘等系统资源,系统负载等数据。
		\item 项目后续的升级和优化:
		\\~ 使用配置文件的方式对监控系统的各参数进行配置,增加系统的健壮性和适用性。
		\\~ 使用例如Zookeeper等分布式协调服务增加分布式的健壮性和适用性。
	\end{itemize}
\end{onehalfspacing}

\datedsubsection{\textbf{基于SMTP协议实现智能邮件}}{2019年4月 -- 2019年5月}
\role{Python, Linux}{个人项目}
\begin{onehalfspacing}
	\begin{itemize}
		\item 项目简介:
		\\~基于 SMTP(Simple Mail Transfer Protocal)简单邮件传输协议,实现的智能邮件,功能包括批量发送,自动回复,自动接收,自动保存附件等。
		\item 功能设计:
		\\~通过smtplib模块实现连接服务器、登录邮箱、发送邮件。通过email模块进行构造邮件(发件人、收件人、主题、正文)等系列操作。设计配置文件,增强程序的实用性。
		\item 使用表现:
		\\~发送邮件时通过读取配置信息自动登录邮箱并发送邮件,且数据稳定、可靠,接受邮件时自动将邮件的主题、内容、附件保存到本地,并将邮箱中的邮件设置为已读状态。
		\item 不足与拓展:
		\\~批量发送邮件功能需依赖脚本,计划增加智能回复功能,通过接受邮件时解析主题进行判断,把需要回复的邮件地址自动保存在发送功能的配置信息中。		
	\end{itemize}
\end{onehalfspacing}
% Reference Test
%\datedsubsection{\textbf{Paper Title\cite{zaharia2012resilient}}}{May. 2015}
%An xxx optimized for xxx\cite{verma2015large}
%\begin{itemize}
%  \item main contribution
%\end{itemize}

\section{\faCogs\ IT 技能}
% increase linespacing [parsep=0.5ex]
\begin{itemize}[parsep=0.5ex]
	\item 开发平台: 熟悉linux平台开发环境,拥有linux平台开发经验。
	\item 编程语言: 熟悉C、C++,了解Python、Bash。
	\item 数据结构: 基本掌握顺序表、链表、栈、队列、哈希表、堆、平衡二叉树(AVL-Tree,SBTree,RBTree)等数据结构。
	\item 算法: 掌握查找(二分、三分、哈希表)、排序(快排、归并、堆排序、桶排序)、搜索(DFS、BFS)、简单动态规划、字符串匹配算法(KMP、Sunday、
Shift-And、字典树及 AC 自动机)等基础算法。
\end{itemize}


\medskip
\section{\faHeartO\ 获奖情况}
\datedline{一等奖,佳木斯大学程序设计竞赛一等奖}{2017年6月}
\datedline{二等奖,第十届黑龙江省蓝桥杯软件组二等奖}{2018年4月}
\datedline{三等奖,第十四届黑龙江省程序大学生程序设计竞赛}{2019年5月}
\medskip

\section{\faInfo\ 其他}
% increase linespacing [parsep=0.5ex]
\begin{itemize}[parsep=0.5ex]
  \item 
  \href{http://bc970321.com}{个人博客:bc970321.com} 
  \item 校科技创新先进个人,单项奖学金、优秀学生干部,三好学生。
  \item 参与筹办第十二届中俄高校数学与计算机竞赛。
  \item 班级班长,校ACM实验室负责人。
\end{itemize}

%% Reference
%\newpage
%\bibliographystyle{IEEETran}
%\bibliography{mycite}
\end{document}
