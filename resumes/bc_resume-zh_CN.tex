% !TEX TS-program = xelatex
% !TEX encoding = UTF-8 Unicode
% !Mode:: "TeX:UTF-8"

\documentclass{resume}
\usepackage{zh_CN-Adobefonts_external} % Simplified Chinese Support using external fonts (./fonts/zh_CN-Adobe/)
% \usepackage{NotoSansSC_external}
% \usepackage{NotoSerifCJKsc_external}
% \usepackage{zh_CN-Adobefonts_internal} % Simplified Chinese Support using system fonts
\usepackage{linespacing_fix} % disable extra space before next section
\usepackage{cite}

\begin{document}
\pagenumbering{gobble} % suppress displaying page number

\name{毕成}

\basicInfo{
	\email{bc970321@163.com} \textperiodcentered\ 
	\phone{(+86) 15765304024} \textperiodcentered\ 
	\github[bc970321]{https://github.com/bc970321}
}

\section{\faGraduationCap\ 教育背景}
\datedsubsection{\textbf{佳木斯大学}, 佳木斯}{2016 -- 至今}
\textit{在读本科生}\ 计算机科学与技术, 预计 2020 年 7 月毕业

\section{\faUsers\ 项目经历}


\datedsubsection{\textbf{分布式服务器健康监控系统}}{2019年3月 -- 2019年5月}
\role{C, Linux}{个人项目}
\begin{onehalfspacing}
	\begin{itemize}
		\item 项目目标:
		 Master+Client的结构,支持上万Client并发,Client端监控集群硬件信息,CPU,内存,磁盘等资源消耗,用户及进程等系统信息,并按照各个指标进行健康状态分级,将故障和可能的风险对Master报警,并将健康信息发送到Master。
		\item 
		Client端使用多进程提高并发度,通过TCP协议进行安全的数据传输,发送心跳信息和等待Master端的链接请求。考虑服务器故障因素,将进行自我检测服务器心跳。考虑磁盘占用量,将大文件进行压缩,使用UDP与服务器进行警告信息通讯。
		\item 
		Master端使用多线程处理各节点数据,监测在线用户、使用epoll主动心跳,维护链表信息,对掉线的Client端进行删除重连、发送数据请求信息并接收数据。使用IO多路复用,提升系统效率。设置警报端口,及时接受警报信息。
		\item 
		使用系统调用的方法获取各硬件信息,CPU,内存,磁盘等系统资源,系统负载等数据。
		\item 
		使用配置文件的方式对监控系统的各参数进行配置,增加系统的健壮性和可扩展性。
	\end{itemize}
\end{onehalfspacing}

\datedsubsection{\textbf{支持多并发的map实现}}{2019年5月 -- 2019年6月}
\role{C++, Linux}{个人项目}
\begin{onehalfspacing}
	\begin{itemize}
		\item 项目简介:
		\\~
		使用模板封装BRTree实现的支持多并发的map。
		\item 功能:
		\\~
		通过设计iterator对map实现算法和数据结构的分离,提升程序的实用性。实用模板类进行封装实现多类型的增、删、改、查等基本操作,以及map的合并、名次的查找、upper-bound和lower-bound。
		\item 性能:
		\\~
		使用Allocator和调整内存的分配和布局对程序的时间效率进行优化,正确管理new和delete的内存分配和销毁,避免内存碎片的产生、优化内存的利用率。		
	\end{itemize}
\end{onehalfspacing}

% Reference Test
%\datedsubsection{\textbf{Paper Title\cite{zaharia2012resilient}}}{May. 2015}
%An xxx optimized for xxx\cite{verma2015large}
%\begin{itemize}
%  \item main contribution
%\end{itemize}

\section{\faCogs\ IT 技能}
% increase linespacing [parsep=0.5ex]
\begin{itemize}[parsep=0.5ex]
	\item 开发平台: 熟悉linux平台开发环境,拥有linux平台开发经验。
	\item 编程语言: 熟悉C、C++,了解Python、Bash。
	\item 数据结构: 基本掌握顺序表、链表、栈、队列、哈希表、堆、平衡二叉树等数据结构。
	\item 基础算法: 熟悉简单查找算法、排序算法、搜索、字符串匹配算法等基础算法。
\end{itemize}

\section{\faInfo\ 其他}
% increase linespacing [parsep=0.5ex]
\begin{itemize}[parsep=0.5ex]
  \item 校科技创新先进个人,校优秀学生干部,三好学生,校 ACM 实验室负责人。
  \item 参与筹办第十二届中俄高校数学与计算机竞赛。
  \item 第九届蓝桥杯黑龙江省二等奖。
  \item 第十四届ACM-CCPC黑龙江省三等奖
\end{itemize}

%% Reference
%\newpage
%\bibliographystyle{IEEETran}
%\bibliography{mycite}
\end{document}
